\chapter{Introductory Material}
\label{chapterlabel1}
% discuss SPECT and Multimodal 
% SPECT 
% Photon Detection (and collimation)
% Event reconstruction (and Image Recon*) % Recon techniques pg51 Mic.
%  Calibration Techniques
This thesis carries out the evaluation of a \acrshort{SPECT} imaging system. This chapter outlines the basics of \acrshort{SPECT} imaging systems and introduces a novel combined imaging system; the \acrshort{INSERT} scanner. 
\section{SPECT}
The \acrlong{SPECT} system was introduced by \cite{Anger} as a method of retrieving a signal from the detection of gamma photons. \acrshort{SPECT} is used widely as a medical imaging technology to investigate the physiological response of an injected radio-pharmaceutical. A pharmaceutical is chosen by its ability to target a given biological process or organ, this drug is labelled with a gamma emitter. The localised emittion is detected by the \acrshort{SPECT} camera and the resulting signal is reconstructed into a 3D tomographic image. The main components of the \acrshort{SPECT} developed by \cite{Anger and Rosenthal} are a collimator, scintillation crystals and \acrlong{PMT}. The gamma emitted from the patient will travel toward the camera head; they interact with the scintillation crystal producing a light pulse, the \acrshort{PMT} with amplify this pulse and convert it to an electrical signal for processing. The signals are processed using Anger logic, \cite{angerLogic}, however without a geometric context it is impossible to localise the origin of the gamma ray; the collimator acts as a camera aperture by allowing only gamma rays from a chosen direction to enter the crystal.

Photon detection within a \acrshort{SPECT} is governed by many factors. The collimator design is the largest contribution to sensitivity, resolution and \acrlong{FOV} of the system. The choice of collimator must consider a balance between number of photons detected and the quality of localisation. A high resolution will improve localisation but is achieved by reducing the aperture size; reducing the sensitivity and requiring a longer scanning time. The collimator is also chosen with consideration of the desired image; collimator designs such as the pinhole are more suited to small organ imaging. The geometry of the collimator determines how the acquired signal is processed into a final image. The chosen collimator geometry defines the photon path however there are uncertainties to consider such as photons penetrating through the collimator septa. 

The localisation of photons is also determined by the design of the scintillation crystals and \acrshort{PMT}. These components determine performance factors of the detector including efficiency, count rate, spatial and energy resolutions, uniformity, and linearity. The size, position and density of the crystal determines the efficiency of photon detection. A greater efficiency improves the count rate as event are not lost in between scintillation interactions due to dead time. The energy resolution is the detectors ability to distinguish the photopeaks within a spectra. The spatial resolution of a detector defines the quality of spatial information for each event, this is governed by the \acrshort{PMT} geometry and the photons \acrlong{DOI} within the crystal. Uniformity and linearity are important factors in image quality as the spatial response may be unstable depending on time or location.
\section{Multi-modal Imaging} % dont discuss INSERT 
% Multi modal 
Multimodal medical imaging technologies have gained popularity within research and clinical practice for their ability to simultaneously or sequentially acquire data from multiple systems. Individual systems can provide either functional or structural information, however many patients require multiple scans for diagnosis or treatment. The combination of this information is essential in investigations and can be achieved through methods such as spatial and time registration. However, acquiring multiple scans and registering the images takes many hours and is subject to errors such motion. Multimodal systems can solve this by acquiring both structural and functional information within one simultaneous scan, reducing scan time and improving image registration. 

Mulitmodal systems such as \acrlong{SPECT/CT} and \acrlong{PET/MRI} have been widely implemented in the clinic as they combine anatomical information from the \acrshort{CT} and \acrshort{MR}, with the biochemical or metabolic information of \acrshort{SPECT} and \acrshort{PET}. The combination of these systems is achieved through the development of innovative technologies and designs, as well as the improved computational methods. This can prove expensive and challenging as individual systems may not be compatible. \acrshort{PET/MRI} was achieved through the development of \acrshort{MR} compatible detectors. \acrshort{PET/MR} and \acrshort{SPECT/MR} have been highly successful in the clinic as it produces combined structural and functional images, however, \acrshort{CT} introduces additional dose and is not as versatile of \acrshort{MRI}. The development of \acrshort{SPECT/MRI} promises the advantages of \acrshort{PET/MR}, but this has yet to be achieved in the clinic. This thesis outlines the use of the\acrshort{INSERT} scanner; the first clinical simultaneous \acrshort{SPECT/MRI} imaging system. The invention of this scanner was achieved due to the rise of \acrshort{MR} compatible nuclear imaging technologies, a compact system and unique collimator design. 

\begin{figure}[htp]
    \centering
    \includegraphics[width=0.9\textwidth,keepaspectratio]{INSERT_design} % name of file without extension
    \caption{The design for the partial ring INSERT system.} \label{fig:INSERT}
\end{figure}


