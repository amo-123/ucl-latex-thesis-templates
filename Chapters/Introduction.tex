\chapter{Introductory Material}
\label{Introduction}
% discuss SPECT and Multimodal 
% SPECT 
% Photon Detection (and collimation)
% Event reconstruction .Recon techniques pg51 Mic.
%Image Recon 
% Calibration Techniques Deb Pg. 49 

This thesis outlines the evaluation of a \acrshort{SPECT} imaging system, designed for the simultaneous use within a clinical \acrshort{MRI} system. This chapter outlines the basics of \acrshort{SPECT} imaging systems, discusses the use of multi-modality imaging and introduces the novel combined imaging system; the \acrshort{INSERT} scanner. 

\section{SPECT}
The \acrlong{SPECT} system was introduced by \cite{Anger} as a method of retrieving a signal from the detection of gamma photons. \acrshort{SPECT} is used widely as a medical imaging technology to investigate the physiological response of an injected radio-pharmaceutical. A pharmaceutical is chosen by its ability to target a given biological process or organ, this drug is labelled with a gamma emitter. The localised emission is detected by the \acrshort{SPECT} camera and the resulting signal is reconstructed into a 3D tomographic image. The main components of the \acrshort{SPECT} developed by \cite{Anger and Rosenthal} are the collimator, scintillation crystals and \acrlong{PMT}. The gamma rays emitted from the patient will travel toward the camera head; they interact with the scintillation crystal producing a light pulse, the \acrshort{PMT} will amplify this pulse and convert it to an electrical signal for processing. The signals are processed using Anger logic, \cite{angerLogic}, however without a geometric context it is impossible to localise the origin of the gamma ray; the collimator acts as a camera aperture by allowing only gamma rays from a chosen direction to enter the crystal.

Photon detection within a \acrshort{SPECT} is governed by many factors. The collimator design is the largest contribution to sensitivity, resolution and \acrlong{FOV} of the system. The choice of collimator must consider a balance between number of photons detected and the quality of localisation. A high resolution will improve localisation but is achieved by reducing the aperture size; reducing the sensitivity and requiring a longer scanning time. The collimator is also chosen with consideration of the desired image; for example collimator designs such as the pinhole are more suited to small organ imaging. The geometry of the collimator determines how the acquired signal is processed into a final image. The chosen collimator geometry defines the photon path however there are uncertainties to consider such as photons penetrating through the collimator septa. 

The localisation of photons is also determined by the design of the scintillation crystals and \acrshort{PMT}. These components determine performance factors of the detector including efficiency, count rate, spatial and energy resolutions, uniformity, and linearity. The size, position and density of the crystal determines the efficiency of photon detection. A greater efficiency improves the count rate as event are not lost in between scintillation interactions due to dead time. The energy resolution is the detectors ability to distinguish the photopeaks within a spectra. The spatial resolution of a detector defines the quality of spatial information for each event, this is governed by the \acrshort{PMT} geometry and the photons \acrlong{DOI} within the crystal. Uniformity and linearity are important factors in image quality as the spatial response may be unstable depending on time or location.

\section{SPECT Imaging} % Reword
\acrshort{SPECT} imaging systems determine the location of a radioactive tracer within the body from the forward projection of the acquired data. The process of imaging is governed by the solution of the equation \ref{eqn:imgRec}. The value of $x$, the tracer activity, is determined from the acquired data, $Y$, and the system matrix, $S$; subject to a noise term, $\eta$. The solution give a complete image of the object by acquiring projections from several angles. More information from the object is acquired with each addition projection angle. We extract the tracer information from a sinogram; a representation of projection data moving through all covered angles. The reconstructed images are produced through the analytical or iterative solution of \ref{eqn:imgRec}. The projections are created using event reconstruction algorithms. These are based on Anger logic; this thesis focuses on two methods of event reconstruction used to evaluate the projection data acquired from each experiment.

\begin{equation} \label{eqn:imgRec}
        Y = S(X) + \eta 
\end{equation}

\subsection{Event Reconstruction: Centroid Method}
The \acrlong{COM} method, also known as the Centroid method, is a simple reconstruction originating in the first Anger cameras. From Anger logic we know the signal from a scintillation event will decrease further from the interaction point. The Centroid algorithm makes use of this by simply considering a weighted mean of the signal of a given event from each photodetector. 

\begin{equation} \label{eqn:Centroid}
        X_{i} = \frac{\sum^{N}_{j} x_{j} \cdot S_{ij}}{\sum^{N}_{j} S_{ij}} \quad ; \quad  Y_{j} = \frac{\sum^{N}_{j} y_{j} \cdot S_{ij}}{\sum^{N}_{i} S_{ij}}
\end{equation}

The equation \ref{eqn:Centroid} describes the reconstructed event positions $X\_i$ and $Y\_i$ for each event i. The signal $S\_{ij}$ from a given event is summed over each  photodetector j. This weighted sum gives the likely position of an event as the coordinates is calculated toward the centre of the nearest photodetecto, $x\_j$ and $y\_j$. The centroid method is modified to account for the event compression towards the centre of a given detector; this modified centroid method subtracts a common baseline, B, value from each event. The baseline subtraction removes signal not within the region of interest and is chosen based on the detectors field of view (\ref{eqn:ModCent}). 

\begin{equation} \label{eqn:ModCent}
        X_{i} = \frac{\sum^{N}_{j} x_{j} \cdot (S_{ij} - B)}{\sum^{N}_{j} S_{ij}} \quad ; \quad  Y_{j} = \frac{\sum^{N}_{j} y_{j} \cdot (S_{ij} - B)}{\sum^{N}_{i} S_{ij}}
\end{equation}

This method is popular as is provides fast and simple projection data. Although the baseline solves the central compression, the centroid method is subject to several errors. 

Non-uniformity in a given camera will distort the resulting projection. Uniformity and sensitivity are subject to the detector response to relative gain and energy window selection. The difference between photodetectors gain can result in false density distribution across the whole detector. The centroid method is unable to account for this as signal will be heavily weighted by a non-uniform photodetector. 

Linearity of the projection data is subject to spatial distortion within the crystal, the centroid method assumes a constant response throughout the crystal. Although true for the centre of the crystal this assumption results in non-linearity at the detector edges. The result is similar to the central compression from equation \ref{eqn:Centroid}.

These issues can be resolved with correction maps as we will see in the following chapters. However we can improve the event reconstruction through statistical methods and spatial modelling. 

\subsection{Event Reconstruction: Maximum Likelihood}
A statistical method of event reconstruction is able  to produce higher quality projection data which is less susceptible to spatial distortions. The \acrlong{ML} algorithm achieves this by accounting for the random nature of gamma particle interactions and extracting probability models from acquired data.

The \acrshort{ML} algorithm makes use of optical models and the detectors spatial response to calculate the events coordinates and energy. The algorithm optimises a set of model parameters in order to maximise the probability of observing the acquired data from the given model. If we consider our observed data by vector $\Bar{\nu}$, each value is given by detector response to a scintillation event, i.e. number of detected photons or electrical signal, by each channel. The observed data is a result of a series of random events, $V\_i$ (in SPECT imaging this is the transport of gamma rays). The events can be represented by a known probability model, $M(\theta)$, where $\theta$ is the set of optimising parameters. Thus the probability of observing our acquired data given $\theta$ is the joint probability of each independent event $V\_i$ for N total events. 

\begin{equation} \label{eqn:MLProb}
                P(\Bar{\nu}|\theta) = \prod^{N}_{i=i}P(V_i)
\end{equation}

The likelihood of the observed sample, $L(\theta|\Bar{\nu}) = P(\Bar{\nu}|\theta)$, can be maximised to retrieve the unknown parameter which gives the observed data. The standard form for the \acrshort{ML} algorithm takes the log of the maximisation to turn the product into a sum. 

\begin{equation} \label{eqn:ML}
                \hat{\theta} = max\{\sum^{N}_{i=i}log(L(\theta|\Bar{\nu}))\}
\end{equation}


\subsection{Image Reconstruction}
Once the projection data has been reconstructed the final images can be produced from the solution to equation \ref{eqn:imgRec}. Here we briefly describe the iterative approach which is used throughout this thesis. The \acrlong{MLEM} algorithm aims to produce a image model which accounts for the physical interactions involved in \acrshort{SPECT} imaging; optimising the model until it matches the acquired data, \cite{4307558}. Using the \acrshort{ML} logic described above the algorithm carries out a series of forwards and back projections in order to improve a estimated model. The algorithm is composed of two main steps; the estimation stage, and the maximisation stage. The initial estimation can be a uniform image or can be retrieved from a simple back projection of the acquisition data. The estimated model is forward projected to produce a model of the measured projection data. The projection models are then compared with the acquired data. A correction term is derived from this comparison; using the maximum likelihood technique a correction is determined and used to improve out estimation. This is the maximisation step, the result is an improved image when back projection is carried on on the estimation. However this improvement is incremental and so the steps must be repeated; improving the model and creating a image closer to the true object with each iteration.  

\section{Multi-modal Imaging} % dont discuss INSERT 
% Multi modal 
The capabilities of medical imaging continues to improve and expand, however in recent years it has become possible to combine medical imaging systems. Multimodal imaging technologies have gained popularity within research and clinical practice for their ability to simultaneously or sequentially acquire data from multiple systems. Individual systems can provide either functional or structural information, however many patients require multiple scans for diagnosis or treatment. The combination of this information is essential in investigations and can be achieved through methods such as spatial and time registration. However, acquiring multiple scans and registering the images takes many hours and is subject to errors such motion. Multimodal systems can solve this by acquiring both structural and functional information within one simultaneous scan, reducing scan time and improving image registration. 

Mulitmodal systems such as \acrlong{SPECT/CT} and \acrlong{PET/MRI} have been widely implemented in the clinic as they combine anatomical information from the \acrshort{CT} and \acrshort{MR}, with the biochemical or metabolic information of \acrshort{SPECT} and \acrshort{PET}. The combination of these systems is achieved through the development of innovative technologies and designs, as well as the improved computational methods. This can prove expensive and challenging as individual systems may not be compatible. \acrshort{PET/MRI} was achieved through the development of \acrshort{MR} compatible detectors. \acrshort{PET/MR} and \acrshort{SPECT/CT} have been highly successful in the clinic as it produces combined structural and functional images. 

However a sequential \acrshort{SPECT/MRI} clinical system has yet to be established. The combination of \acrshort{SPECT} and \acrshort{MRI} can be carried out through sequential use of \acrshort{SPECT/CT} and \acrshort{MRI}. The data are registered to provide the \acrshort{SPECT/MR} images useful in areas should as tumour treatment planning. In the case of \acrshort{SPECT/CT}, the radiation \acrshort{CT} introduces additional dose and the acquisition is not as versatile of \acrshort{MRI}. The development of simultaneous and integrated \acrshort{SPECT/MRI} promises the advantages of \acrshort{PET/MR}. \acrshort{MRI} has proven a powerful and versatile imaging method, providing greater diagnostic information with no dose. \acrshort{SPECT} imaging is more widely available and cheaper than \acrshort{PET} systems; \acrshort{SPECT} is popular as it allows the use of more accessible radionuclides and dual-isotope imaging.

It stands to reason that a simulataneous clinical \acrshort{SPECT/MR} system could provide a multifaceted imaging tool. This thesis outlines the use of the\acrshort{INSERT} scanner; the first clinical simultaneous \acrshort{SPECT/MRI} imaging system. The goal of which is to provide brain tumour imaging and treatment stratification, and establish a foundation of a completely integrated \acrshort{SPECT/MRI} system. 

\begin{figure}[htp]
    \centering
    \includegraphics[width=0.9\textwidth,keepaspectratio]{INSERT_design} % name of file without extension
    \caption{The design for the partial ring INSERT system.} \label{fig:INSERT}
\end{figure}

