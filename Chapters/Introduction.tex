\chapter{Introductory Material}
\label{chapterlabel1}

\section{Multi-modal Imaging} % dont discuss INSERT 
% discuss SPECT and Multimodal 
% SPECT 
% Photon Detection (and collimation)
% Event reconstruction (and Image Recon*) % Recon techniques pg51 Mic.
%  Calibration Techniques
% Multi modal 
Multimodal medical imaging technologies have gained popularity within research and clinical practice for their ability to simultaneously or sequentially acquire data from multiple systems. Individual systems can provide either functional or structural information, however many patients require multiple scans for diagnosis or treatment. The combination of this information is essential in investigations and can be achieved through methods such as spatial and time registration. However, acquiring multiple scans and registering the images takes many hours and is subject to errors such motion. Multimodal systems can solve this by acquiring both structural and functional information within one simultaneous scan, reducing scan time and improving image registration. 

Mulitmodal systems such as SPECT/CT and PET/MRI have been widely implemented in the clinic as they combine anatomical information from the CT and MR, with the biochemical or metabolic information of SPECT and PET. The combination of these systems is achieved through the development of innovative technologies and designs, as well as the improved computational methods. This can prove expensive and challenging as individual systems may not be compatible. 

\paragraph{}
Dual imaging has seen the development of simultaneous \acrshort{SPECT/CT} and \acrshort{PET/MR} scanners. \acrshort{PET/MR} has lead to the development of \acrshort{MR} compatible imaging technologies; the combination of nuclear medicine and \acrshort{MR} provides improved diagnostic information with fewer scans. Despite the uptake in clinical  \acrshort{PET/MR} there is yet to be use of a clinical \acrshort{SPECT/MR} system. The rise of \acrshort{MR} compatible nuclear imaging technologies has lead to the development of the first clinical \acrshort{SPECT/MR} scanner.   
\paragraph{}
This thesis outlines the use of the\acrshort{INSERT} scanner; the first clinical simultaneous \acrshort{SPECT/MRI} imaging system. \acrshort{SPECT/MR} has had success in pre-clinical systems, but has faced challenges with development of a clinical scanner; the \acrshort{INSERT} has overcome these challenges with its compact system and unique collimator design. 

\begin{figure}[htp]
    \centering
    \includegraphics[width=0.9\textwidth,keepaspectratio]{INSERT_design} % name of file without extension
    \caption{The design for the partial ring INSERT system.} \label{fig:INSERT}
\end{figure}

\section{Objectives}
The INSERT project promises to lay the ground work for a full body SPECT/MRI system. Such a system holds many advantages over PET/MRI, as SPECT is the more widely used and can image at a larger range of radioisotopes \cite{doi:10.1177/153303460600500406}. However to further the development of INSERT, thorough testing must be carried out to ensure it is ready for clinical use. The focus of this thesis is to evaluate the potential performance of the INSERT design for different radionuclide, with a focus on the impact of septal penetration and depth of interaction (DOI). The INSERT system has been constructed in a simulated environment using Monte Carlo software \cite{1236960}. The system was tested by running experiments in the simulation. The simulated experiment allows the source and scanner properties to be defined. The novel collimator has been simulated and the experiments focus on testing its robustness over a range of energies. Recording the interactions within the scanner will determine how effectively the collimator is able to direct photons without degrading image quality.
\paragraph{} 
The quality of the images can be reduced if the system is susceptible to septal penetration. This is the penetration of photons through the collimator's septa, causing an degradation in the collimator resolution \cite{0031-9155-18-6-005}, \cite{0031-9155-50-21-004}. Septal penetration was measured by carrying out experiments with various phantom types and measuring the effects of high energy photons on the resulting data in order to model the extent of penetration in the collimator. The DOI determines how far the photons travels through the scintillation crystal before it interacts \cite{0031-9155-55-2-N04}. The DOI is required when reconstructing the images acquired from the scan. Accurately predicting the DOI improves the performance of the image reconstruction algorithm. The interaction data recorded from the experiments allows the penetration and DOI to be estimated. Moreover the reconstructed images from the experiment allows the impact of these phenomena to be realised in the final images.

\section{Motivation}
The development of a SPECT/MRI will provide many advantages due to the diverse applications that SPECT imaging provides. The use of PET/MRI is limited to a single radionuclide energy. SPECT is widely used as a range of radionuclide energies can be used for dual-radionuclide imaging. The use of simultaneous SPECT/MRI aims to improve treatment for brain tumours and radiotherapy planning \cite{doi:10.1259/bjr.20160690}. The combined modalities reduces the number of scans a patient will have and so provides a more comfortable treatment. The INSERT aims to determine functional information from scans of known tumours in order to track development and plan treatments. Due to the novel design of the SPECT insert, it is essential that the component properties are analysed and the system is tested and classified. It is important to test the fundamental principles governing the system and determine how the device will respond in use.
\paragraph{}
In order to carry out thorough testing the SPECT, insert was constructed in a Monte Carlo simulation. This simulated environment allowed for complete control of the system and the experimental design. The testing carried out can record all interaction with the scanner, providing information that cannot be achieved easily through the physical scanner. The simulations allows for safer and cheaper testing of the scanner design. The experiments are diverse and easily changed, the scanner design can also be changed between experiments. The flexibility of the simulated experiments provides many advantages over testing the physical system. The results from the simulations can help define a quantified model used to aid future testing and system design improvements. Accounting for septal penetration and DOI in the physical scanner requires the development of a model for these phenomena that will predict its impact, the Monte Carlo experiment can be used to test these models. 
\blindtext
