% I may change the way this is done in a future version, 
%  but given that some people needed it, if you need a different degree title 
%  (e.g. Master of Science, Master in Science, Master of Arts, etc)
%  uncomment the following 3 lines and set as appropriate (this *has* to be before \maketitle)
% \makeatletter
% \renewcommand {\@degree@string} {Master of Things}
% \makeatother

\title{A Thesis Title}
\author{Ashley James Morahan}
\department{Department of Medical Physics and Biomedical Engineering}

\maketitle
\makedeclaration

\begin{abstract} % 300 word limit
The aim of this project was to assess the performance of a novel Single Photon Emission Computed Tomography system (SPECT); designed to be compatible with Magnetic Resonance Imaging (MRI). The SPECT insert is the first clinical system of its kind, aiming to improve patient treatment and perform simultaneous scanning of the brain. The system has been created in a simulation and testing of its design was carried out. A series of experiments were designed using different system geometry, phantoms and photon energy. The aims of these experiments were to evaluate the system performance under a number of conditions, pushing the system past the functional environment it was designed for.
\paragraph{}
The novel system required thorough testing in order to validate its design specifications and quantify its performance. A focus of the testing was to determine the extent of septal penetration through the novel collimator design. We aimed to measure this phenomenon and use a model to quantify its effects on the system. Through this result we hoped to determine the limits of the system and to improve the design. Another focus was to measure the depth of interaction of high energy photons. The result of this measurement could improve the image reconstruction algorithm, allowing for higher quality clinical images to be produced.
\paragraph{}
The results of the simulated experiments helped to determine the limits of the systems performance. Analysing the system built an understanding of the novel design and can help to improve the system. The information gained from these experiments can be used to improve the design and work towards a fully operational dual-imaging system.
\paragraph{}
The development of the first clinical simultaneous Single Photon Emission Computed Tomography (SPECT) and Magnetic Resonance Imaging (MRI) system was carried out within the INSERT project. The INSERT scanner was constructed under the initial project but its performance was not fully evaluated; here we have reconstructed the first images on the SPECT system. Calibration and acquisition protocols were developed and used to establish the clinical feasibility of the system. The image reconstruction procedures were implemented on the first phantom images in order to assess the system's imaging capabilities. This study solved issues involving incomplete data sets and pixel failure in the prototype detector system. The final images determined a measure of trans-axial image resolution, giving average values of 9.14 mm and 6.75 mm in the radial and tangential directions respectively. The work carried out on the complete system produced a number of clinical phantom images which utilised the capabilities of both SPECT and MRI.
\end{abstract}



\begin{acknowledgements}
I acknowledge the use of simulations designed by Dr. D\'ebora Salvado, \cite{phdthesis}, and image reconstruction provided by Dr. Kjell Erlandsson, \cite{8069508}.
\paragraph{}
I would like to thank Professor Brian Hutton and Dr. Kjell Erlandsson for providing their advice and supervision in this project. 
\paragraph{}
Thank you to the team at INM and the UCL CDT for all your support this year. 
\end{acknowledgements}

\setcounter{tocdepth}{2} 
% Setting this higher means you get contents entries for
%  more minor section headers.

\tableofcontents
\listoffigures
\listoftables
%\newacronym{spect}{SPECT}{Single Photon Emission Computed Tomography}

\newacronym{mri}{MRI}{Magnetic Resonance Imaging}
%\printglossary[type=\acronymtype]
