% I may change the way this is done in a future version, 
%  but given that some people needed it, if you need a different degree title 
%  (e.g. Master of Science, Master in Science, Master of Arts, etc)
%  uncomment the following 3 lines and set as appropriate (this *has* to be before \maketitle)
% \makeatletter
% \renewcommand {\@degree@string} {Master of Things}
% \makeatother

\title{Evaluation and Implementation of Clinical SPECT/MRI System}
\author{Ashley James Morahan}
\department{Department of Medical Physics and Biomedical Engineering}

\maketitle
\makedeclaration

\begin{abstract} % 300 word limit
The aim of this thesis was to evaluate the performance of a novel \acrshort{SPECT} system, designed for simultaneous use within a clinical \acrshort{MRI}. The scanner known as the \acrshort{INSERT} is the first clinical system of its kind, developed to provide improved tumour treatment stratification and establish the ground work for clinical \acrshort{SPECT/MRI}. The scanner imaging performance has not been fully evaluated; in this work we have acquired and reconstructed the first images on the novel system. Calibration and acquisition protocols have been implemented and assessed here to establish the clinical feasibility of the system. The image reconstruction procedures were implemented on the first phantom images in order to assess the system's imaging capabilities. This thesis tackled issues involving detector instability, causing incomplete data sets and pixel failure in the prototype detector system. A study on the system's stability established the need for new calibration protocols and software. The development of this software aims to provide a multifaceted tool to carry out successful acquisitions on this \acrshort{INSERT}. A result of the first extensive use of the system a number of acquisition protocols and reconstruction methods were developed.  This work laid the ground work for the first use of the complete system which utilises the capabilities of both \acrshort{SPECT} and \acrshort{MRI}. Following this study we aim to implement the \acrshort{INSERT} within an \acrshort{MRI} and establish its use as a complete clinical system. 
\end{abstract}

\begin{acknowledgements}
I acknowledge the use of image reconstruction provided by Dr. Kjell Erlandsson \cite{8069508}.
\paragraph{}
I would like to thank Professor Brian Hutton and Dr. Kjell Erlandsson for providing their advice and supervision in this project. 
\paragraph{}
Thank you to the team at INM and the UCL i4Health for all their support. 
\end{acknowledgements}

\setcounter{tocdepth}{2} 
% Setting this higher means you get contents entries for
%  more minor section headers.

\tableofcontents
\listoffigures
 \listoftables
%\newacronym{spect}{SPECT}{Single Photon Emission Computed Tomography}

\newacronym{mri}{MRI}{Magnetic Resonance Imaging}
\printglossaries
