\chapter{Experiments 01}
\label{chapterlabel}


% Evaluation method and calibration 
% Detector Cal 
% System Cal 
% System evaluation 
% hard ware issues 

% New Event Recon (killed_LRF, FilteredSpectrum, Normalisation)
% New Linearity software (3 methods) 
In this chapter the use of the \acrshort{INSERT} scanner is outlined. The calibration procedures and experimental methods explored in this chapter are used to evaluated the system performance. The outcome of this investigation results in the implementation of new data processing methods. 

\section{Introduction}
The following experiments were carried out at San Raffaele Hospital (Milan), in collaboration with Politecnico di Milano (POLIMI). The system was transported and installed in the nuclear medicine department in order to carry out experiments with radioisotopes such as Technetium-99m. The experiments were carried out using the BULMA software provided by Mediso Ltd (Budapest). The software was designed to operate the system and acquired the raw data from experiments. The raw data was processed and calibrated using the BULMA INSERT$_$GUI package along with existing system data. 

This experiments set out to acquire the first phantom images from the \acrshort{INSERT} system, in order to achieve this the system was first evaluated and calibrated. The count-rate and energy resolution were determined and the system corrections and calibrations for these parameters where carried out at Polimi. The calibration procedures required to correct for event and image reconstruction are out outlined here.

The calibration procedure was developed on a single detector head system \cite{DebCal}, and was tested in this investigation for the first time on the complete \acrshort{INSERT} scanner. This investigation aimed to determine the effectiveness of the proposed calibration method, and evaluate the systems performance and imaging capabilities. The calibration involves two stages, the detector calibration and the system calibration. The event reconstruction is corrected through the detector calibration procedure; a standard linearity and uniformity correction is performed. The image reconstruction is carried out using the geometric and sensitivity information obtained by the system calibration.

\subsection{Detector Calibration}
The detector calibration is carried out by collecting uniformity and linearity data from each detector head. The uniformity data is collected by performing a flood scan of a single point source located in the centre of the scanner\acrshort{FOV}. This scan is carried out with the collimator attached; short scans are required due to the high counts with no collimation. This scanned was carried out on the first day as a quick check of detector coverage and performance. The scans are used to correct for uniformity in the subsequent phantom acquisitions. 

As discussed previously the planar event reconstruction is carried out in a two step method; a centroid reconstruction is used to initialise an \acrshort{MLEM} reconstruction. Part of the \acrshort{MLEM} algorithm is the modelling of the detectors spatial response, known as the \acrshort{LRF}. The BULMA software included existing \acrshort{LRF} data for the system, however it was found that this data was incorrect for the current system. The uniformity flood was used to produce a new set of \acrshort{LRF} models. 

The uniformity data was used to generate a model the for
Light Response Function (LRF) of each of the 72 channels
in the SiPM readouts. These models were used along side the
uniformity flood to generate the system correction parameters.

%\subsection{Uniformity}
%\subsection{Linearity} 
\subsection{System Calibration}
\subsection{Geometry}
\subsection{Sensitivity}
The INSERT project has overseen the development
of the first clinical simultaneous Single Photon Emission Computed
Tomography (SPECT) and Magnetic Resonance Imaging
(MRI) system. In this work we have for the first time implemented
a novel calibration procedure and acquired tomographic images
on the complete clinical system. The calibration procedure was
carried out here on partial ring system, where it has previously
been developed on a single detector. The complete calibration
allowed us to characterise the system as well as parameterise
the image reconstruction algorithm. The novel system design
requires correction parameters by characterising the uniformity,
linearity, and geometry of the system. We were able to carry
out the tomographic imaging of several phantoms including,
cylindrical, simplified Jaszczak, and Hoffman phantoms. During
the reconstruction and processing of the imaging data we were
able to overcome hardware and software issue which were
identified in the characterisation stage. Here we were able to
solve issue involving incomplete data sets and instabilities in
detectors. The work carried out on the complete system allowed
us to evaluate the feasibility of clinical use of this system, we
were able to improve aspects of the calibration procedure and
overcome seeming inherit limitations in the partial ring system.

INSERT sets out to establish the worlds first clinical simultaneous
SPECT MRI system. The SPECT insert is
designed for complete integration with any standard clinical
MRI system, allowing for the simultaneous acquisition of brain
images.The stationary insert system is composed of a partial
ring of 20 CsI:TI scintillation detectors, each with a silicon
photomultiplier (SiPM) readout. The use of the multimini slitslat
(MSS), [1], collimator requires the implementation of a
novel calibration procedure which was established on a single
detector system, [2]. Though proven successful on a single
detector, it was essential to verify the methods on a complete
partial ring system. Carrying out the calibration required
modifications to the single detector methods; 20 detectors had
to be calibrated individually and the ring geometry of the
system introduce space restrictions. Here we have illustrated
the changes carried out in the transition from a single detector to a complete detector system, outlining the steps required and
the challenges overcome in our experiments. We also discuss
the initial results of our extensive tomographic acquisitions
and the methods involved in the image reconstruction.
Development of the clinical insert established the MSS collimator,
[1], this design requires the use of a novel calibration
and image reconstruction. The calibration procedure generates
a set of correction parameters from the system, which determine
the systems uniformity, linearity, and geometry. In this
work we have outlined the acquisition and application of these
parameters. These acquisition corrections were applied with a
software provided by Mediso; producing a set of projection
images to be constructed.
As can be expected from the initial testing of a prototype
the system had issues with both hardware and software during
this research. We have been able to identify these issue,
implement corrections to overcome them and plan for future
improvements to the system and our experimental method. The
hardware issues reside in an instability in the detector modules.
Each SiPM readout is comprised of 72 channels, of which 2
of the total 1,440 produced the incorrect signal and had to be
switched off during most acquisitions. To correct this we were
able to amend the Mediso software and produce a correction
algorithm which was able to interpolate the missing data. This
provided a powerful means of producing projections despite
missing data. In future a more stable solution would be to
replace the readouts. This example demonstrates the kind of
issues encountered in this work; due to inexperience with a
novel system, requiring retrospective software solutions, and
solved in future with final hardware solution.
The final tomographic images were subject to a number
of corrections and post processing, some were inherit to the
system design where others were subject to our experimental
method. Despite this, we were able to extract the desired
images parameters and produce a set of phantom acquisitions
as a proof of concept.

\section{Methods}
Here we discuss the methods used to acquire the relevant
data and apply it to our image reconstruction procedure.
The image reconstruction algorithm follows a standard
Maximum Likelihood Estimation Maximisation (MLEM) algorithm,
however, the INSERT system requires the aforementioned
system corrections and resolution modelling to account for the MSS geometry. The following sections outline
the methods required for image reconstruction and how we
adapted them for the complete 20 detector system.

\subsection{Initial System Testing}
\subsection{Count-rate} 
\subsection{Energy}
\subsection{Resolution}
\subsection{Detector Calibration}
\subsection{Uniformity}


\subsection{Linearity} 
\subsection{System Calibration}
\subsection{Geometry}
The geometric calibration procedure was required to account
for the MSS geometry. The full ring procedure was not
changed from the previous design, [2]. The MSS collimator
is installed and a set of vertical and horizontal capillaries
are scanned. The acquisitions from set of capillary sources
are corrected for linearity and uniformity with the previous
procedures. The resulting images were used to determine
a set of Twist and Tilt parameters required in the image
reconstruction. The Twist and Tilt parameters determined the
angles of distortion in the axial direction and trans-axial
plane respectively. The angles were calculated by a measure
of relative position of the capillary sources and modelling
the distance as a function of the radial position. Extracting
these parameters allowed us to correct the distortion in the
projections caused by the collimator geometry.
The capillary sources served a second purpose; by combining
the acquisitions we produced an image of 61 capillary
sources spanning 5 radial positions. This image was used to
determine the image resolution of the system subject to radial
position. By modelling the capillaries in a the trans-axial plane
with a 2D Gaussian, we determined the radial and tangential
image resolutions.

\subsection{Sensitivity}
\subsection{Phantom Acquisitions}
\subsection{Capillary}
\subsection{Cylinder}
\subsection{Cold Rods}
\subsection{Hot Spheres} 
\subsection{Hoffman}

\section{Results}
\subsection{Initial System Testing}
\subsection{Count-rate} 
\subsection{Energy}
\subsection{Resolution}
\subsection{Detector Calibration}
\subsection{Uniformity}
\subsection{Linearity} 
\subsection{System Calibration}
\subsection{Geometry}
\subsection{Sensitivity}
\subsection{Phantom Acquisitions}
\subsection{Capillary}
\subsection{Cylinder}
\subsection{Cold Rods}
\subsection{Hot Spheres} 
\subsection{Hoffman}

\section{Discussion}

\section{Conclusion}