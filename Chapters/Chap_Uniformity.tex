\chapter{Uniformity and Stability}
\label{Linearity}
% Uniformity Study 
% New Event recon 
% Linearity Software 

\section{Introduction}
As discussed in chapter \ref{Milan} the experiments demonstrated an instability in the detectors uniformity. We set out to determine the extent of this instability and by investigating the uniformity of the detectors. We look to implement protocol to account for detector failure and produce correction maps.

The investigations carried out in chapter \ref{Milan} were all carried out in \acrshort{HSR}. The work presented here was carried out at \acrlong{UCH} after the system was transported and installed in the nuclear medicine department. Due to the change in environment and transportation the system required a diagnostic test to ensure the system was performing well. The system was found to be functional, however there were changes of note; the new environment had a high background radiation count than the \acrshort{HSR} lab, the temperature was also higher so the cooling system was less stable. 

\section{Uniformity}
The experiments carried out in \acrshort{HSR} were time constrained and so only 2 measures of uniformity were carried out. In this investigation we included uniformity from experiments carried out in \acrshort{UCH}. A total of 41 measurements were collected over 6 months on 18 dates. Each uniformity scan required the collimator to be removed, with a point source was placed in the centre of the bore.  
\subsection{Methods}
A measure of uniformity with each detector head was determined through the NEMA quality control procedures for clinical gamma cameras. The \acrshort{NEMA} QC on a gamma camera which has pixel size $6.4 mm \pm 30\%$, is carried out with a flood scan of 10000 counts in the \acrlong{UFOV} (or \acrlong{CFOV}). To normalise the uniformity in each acquisition we only consider the first 5 million counts in detector. We looked at uniformity in the full \acrshort{FOV} and the \acrshort{UFOV}; taking measurements of integral uniformity and the \acrlong{COV} for each detector. 

Integral uniformity is a measure of global uniformity over the whole planar projection. The maximum and minimum count (C\_{max} and C\_{min} respectively) of each detector is calculated. Equation \ref{eqn:IntUni} is used to calculate the integral uniformity as a percentage; the lower the value the greater the uniformity. This however is limited by only considering two values from each detector. 

\begin{equation} \label{eqn:IntUni}
        U_I = (\frac{C_{max} - C_{min}}{C_{max} + C_{min}})\times100\%
\end{equation}

The \acrshort{COV} is a measure of uniformity which considered all values within a detector. Equation \ref{eqn:CoV} determines the \acrshort{COV} as a percentage of standard deviation (SD) over the mean ($\mu$) of all counts in a given detector. The lower the \acrshort{COV} the greater the uniformity. 

\begin{equation} \label{eqn:CoV}
        CoV = \frac{SD}{\mu}\times100\%
\end{equation}

From the 41 scans we considered the $U_I$ and \acrshort{COV} over the whole detector \acrshort{FOV} and \acrshort{UFOV}. The $U_I$ was used to determine detector failure for each day. As the flood source was located in the centre of all detectors,the system uniformity (figure \ref{fig:systemUI}) was calculated using all counts. Failed detectors were identified by values of $ U_{system} > 70\%$; the channel causing the failure was identified by anomalous counts. Figure \ref{fig:RawUni} shows the raw data from a single scan on each day; the matrices show the nodes on each row and channels in each column. These matrices show how the faulty channels reduce uniformity across the system. After removing the faulty channels (figure \ref{fig:HealthUni}) we check $U_{system}$ (figure \ref{fig:fixsystem} and recalculate the $U_I$ and \acrshort{COV}.

\subsection{Results}

We first consider the raw uniformity data. Figures \ref{fig:RawUBox} and \ref{fig:RawCovBox} show box plots of the uniformity study. Faulty detector are immediately clear by the high variation and poor uniformity. The \acrshort{UFOV} improves the uniformity slightly as the edge of the detectors have greater distortion and are not considered in reconstruction. However the stability of the channels is independent of the \acrshort{FOV} and so figures \ref{fig:UFOVUBox} and \ref{fig:UFOVCovBox} shows little improvement in detector uniformity. We must consider healthy channels to gauge uniformity of the detectors.

Figure \ref{fig:HealthUI} shows $U_I$ for the healthy channels. The presence of a faulty channel did not define poor uniformity as many detectors with few faults have lower $U_I$ than those with many faults. However detectors, such as 13, which had faults for large periods of time gave a wide spread of values. Detectors was consistently poor uniformity suggest the presence of undetected faults. 
The integral uniformity is not an effective measure for detectors with faults and high variability as it only considers the extreme values. We consider the \acrshort{COV} for the healthy channels (figure \ref{fig:HealthCOV}). This has a smaller spread due to detector failure as shown in detector 13, suggesting the remaining healthy channels are more uniform. However faulty detectors have fewer channels so will present less variation. 

Finally the \acrshort{UFOV} was considered over the healthy channels. From figure \ref{fig:healthUFOVU} and \ref{fig:healthUFOVCOV} it is clear that the unstable channels can still demonstrate a good uniformity once the corrections are applied. The uniformity between detectors are independent as they act as separate units, however they share a geometric relation. As stated the scanning room had a high background count which was subject to position; if one side of the detectors are closer to a source then it stands to reason that these detectors will share similar non-uniformity due to the source proximity. 

\subsection{Discussion}
The integral uniformity gave a good measure to identify faulty channels and determine the need for manual intervention. We look to establish a standard baseline in which to measure future uniformity. With more scans we can determine a suitable value of $U_I$ which will give an acceptable level of detector uniformity. The \acrshort{COV} values over all scans gives a measure of detector stability. The \acrshort{COV} of a given detector over the 41 scans revealed how stable the uniformity was within the 6 months. We can use this to diagnose the health of the detector and make decisions replacement or fixes. 

Figure \ref{fig:Faulty} reveals key dates in the systems use. The initial scans in \acrshort{HSR} (06/03/2019 to 15/03/2019) show the degradation in the system after 2 weeks of constant use. The following scans took place after the \acrshort{UCH} installation, after a initial testing (31/07/2019) we opened the system and check each detector connection. It was found that large detector failure could likely to be simply caused be loose connections of the ASIC board, this could be caused by moving the system or the cooling process. However this was not consistent as some detectors stabilised without manual intervention. The system was opened again (01/10/2019) to fix the ongoing issues with detector 13 and 14. 

Instability within the same day reveals the need for regular monitoring of the system. Further investigation is required to determine the cause of these faults. We currently determine the best match of acquisition data to the uniform scan in order to apply corrects for that day, following this study we must be aware of inconsistency between consecutive scans and account for detector failure. Each scan is checked after acquisition to ensure a success, if hardware or software failure occurs the scan is repeated. We aim to implement a standard measure to identify this failure. 

This investigation was limited by the practicality of acquiring regular uniformity scans. As the scans required the collimator to be removed (a timely and difficult procedure) we were limited to two days of uniformity scans in \acrshort{HSR}. When the system was transferred to \acrshort{UCH} we had more time to carry out uniformity scans, however it was only after the implementation of a collimator loading system that daily floods could be acquired. 

Future uniformity studies could be improved by recording external conditions. By recording temperature and background radiation we can attempt to identify the source of instability in the detectors. The \acrshort{UCH} experiments were limited by the high background count in the scanning room. Although shielded the background sources varied daily, it was not found to significantly effect image quality as the image processing and energy windowing will remove this, however for this study we used the list mode data and so was more susceptible to background and faulty channels. 

We must also address the definition of channel failure. During acquisition channel failure was defined by a high false signal which spilled into neighbouring channels, however we extended the definition for this study. Channel failure included any channel which posed a significantly high or low count in the raw data. Both of these types of failure can be corrected for in our calibration and reconstruction procedure. We have outlined previously the use of \acrshort{LRF} and uniformity correction can be used to correct acquisition data. Another definition of channel failure was the whole detector failure, this was not due to a signal channel and so could not be corrected for through \acrshort{LRF}. Although manual intervention can solve detector failure, we cannot carry this out when the system is in use; dual acquisition covered the data loss in the event of detector failure and we can capture the missing detector angles within a rotated position. As we can correct for these faults we are still able to produce images in the presents of unstable detector uniformity. 

\section{Conclusion}
The results of the uniformity study highlighted the importance of quality control during acquisition and event reconstruction. As we continue to carry out experiments we set to monitor the uniformity and stability of the detectors more regularly, so that corrections or repeat measurements can be carried out.

During these investigations it was found that the event reconstruction could be improved by introducing a normalisation of the data in order to improve uniformity. Normalising the data allows uniformity and \acrshort{LRF} corrections to be carried out easily on faulty detectors. As we discussed, the raw data reveals  non-uniformity due to instability which can be accounted for during event reconstruction. 

The instability of the detectors lead to many repeat measurements which proved time consuming. The need for a calibration procedure that can be quickly reproduced is essential to carry out the necessary corrections. We look to develop a calibration software that can account for new protocols. 
